RESUMO DO SEU TRABALHO EM INGLÊS. Although it is usually brief (typically 150-300 words), an abstract is an important part of journal article writing (as well as for your thesis and for conferences). Done well, the abstract should create enough reader interest that readers will want to read more! 
Whereas the purpose of an introduction is to broadly introduce your topic and your key message, the purpose of an abstract is to give an overview of your entire project, in particular its findings and contribution to the field. An abstract should be a standalone summary of your paper, which readers can use to decide whether it's relevant to them before they dive in to read the paper. 
[1] Usually an abstract includes the following: 
(a) A brief introduction to the topic that you're investigating.
(b) Explanation of why the topic is important in your field/s.
(c) Statement about what the gap is in the research.
(d) Your research question/s / aim/s.
(e) An indication of your research methods and approach.
(f) Your key message.
(g) A summary of your key findings.
(h) An explanation of why your findings and key message contribute to the field/s.
[2]  In other words, an abstract includes points covering these questions.
(a) What is your paper about?
(b) Why is it important?
(c) How did you do it?
(d) What did you find?
(e) Why are your findings important?
To see the specific conventions in your field/s, have a look at the structure of a variety of abstracts from relevant journal articles. Do they include the same kinds of information as listed above? What structure do they follow? You can model your own abstract on these conventions. (https://www.anu.edu.au/students/academic-skills/research-writing/journal-article-writing/writing-an-abstract)

\begin{keywords}
palavra1, palavra2, palavra3, todas em inglês
\end{keywords}